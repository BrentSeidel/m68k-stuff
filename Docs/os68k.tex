\documentclass[10pt]{book}
%
\usepackage{url}
%
%  Include the listings package
%
\usepackage{listings}
%
%  Define Tiny Lisp based on Common Lisp
%
\lstdefinelanguage[Tiny]{Lisp}[]{Lisp}{morekeywords=[13]{atomp, bit-vector-p, car, cdr, char-downcase, char-code, char-upcase, compiled-function-p, dowhile, dump, exit, fresh-line, if, code-char, lambda, msg, nullp, parse-integer, peek8, peek16, peek32, poke8, poke16, poke32, progn, quote, read-line, reset, setq, simple-bit-vector-p, simple-string-p, simple-vector-p, string-downcase, string-upcase}}
\lstset{language={[Motorola68k]Assembler}}
%
% Macro definitions
%
\newcommand{\operation}[1]{\textbf{\texttt{#1}}}
\newcommand{\package}[1]{\texttt{#1}}
\newcommand{\function}[1]{\texttt{#1}}
\newcommand{\constant}[1]{\emph{\texttt{#1}}}
\newcommand{\keyword}[1]{\texttt{#1}}
\newcommand{\datatype}[1]{\texttt{#1}}

%
% Front Matter
%
\title{Notes and Background on Operating System for the 68000 CPU}
\author{Brent Seidel \\ Phoenix, AZ}
\date{ \today }
%========================================================
%%% BEGIN DOCUMENT
\begin{document}
\maketitle
\begin{center}
This document is \copyright 2024 Brent Seidel.  All rights reserved.

\paragraph{}Note that this is a draft version and not the final version for publication.
\end{center}
\tableofcontents

%========================================================
\chapter{Overview}

This is a collection of notes on a simple multi-tasking operating system for the 68000 CPU, temporarily names os68k.  The main goals are:
\begin{enumerate}
  \item To be able to blink lights in interesting pattern in the Pi-Mainframe (\url{https://github.com/BrentSeidel/Pi-Mainframe}) project,
  \item To learn something about operating system design, and
  \item To actually have a somewhat useful operating system.
\end{enumerate}

Note that all statements are subject to change as the project develops.

This is based on the 68000 simulator part of the Sim-CPU (\url{https://github.com/BrentSeidel/Sim-CPU}) project.  It is targeted towards a system with 16Mbytes of memory and no MMU.  The memory is divided into 16 one megabyte sections, the first (lowest) section is for the operating system and each task gets one section.  The gives a system that supports up to 15 user tasks plus the operating system.

Simulated devices provided by Sim-CPU that are currently used are the clock to provide a periodic interrupt for tasking and the serial-telnet port for console I/O.  Simulated disks are not yet supported.

The operating system also includes a library with utility routines that can be used by the user programs.

%--------------------------------------------------------------------------------------------------
\section{Kernel}
The kernel is composed of several sections.  The first two are at locations that are fixed by the simulated CPU and hardware:
\begin{enumerate}
  \item The CPU vector table starts at address 0 and contains 256 long word entries.  This occupies 1 kilobyte of space.
  \item The I/O port section runs from the end of the vector table at an address of 400$_{16}$ and runs to 1000$_{16}$.
\end{enumerate}
The remaining sections are arbitrarily arranged and will probably change with development.
\begin{enumerate}
  \item HW\_SECT contains code for interfacing with the hardware devices and related interrupt service routines.
  \item OS\_SECT contains the main operating system code including initialization, the clock interrupt, context save/restore, scheduling, exceptions, and system calls.
  \item OS\_DATA contains the operating system data tables and messages.  The data tables currently include the task control blocks and the console device blocks.
  \item LIB\_SECT contains library routines for use by the operating system and user programs.  Note that routines that use system calls shouldn't be used by the operating system, at least not yet.
  \item LIB\_DATA contains library data.  Note that since library routines may be in use by multiple tasks simultaneously, this should be constant data only.  Any variable data should be allocated on the stack.
  \item OS\_STACK is space for the operating system stack.
  \item USR\_STACK is space for the operating system user stack.  The is used by the null task that runs when no other task can be run.  Its stack needs are minimal.
\end{enumerate}

\subsection{Hardware Abstraction Layer}
The currently supported hardware includes a clock that provides periodic interrupts and a terminal interface that can be accessed externally by telnet (or gtelnet).

\subsubsection{Clock}
The clock has a settable rate for the periodic interrupts.  The rate is a byte size port which allows any value 0-255.  The multiplier in the simulation is 100mS, thus to get a 1 second interrupt, the rate would be set to 10.  Setting the rate to one gives a 100mS interrupt (or 10 times a second).  This is a reasonable rate for multitasking.  As some point, the simulation may get adjusted to allow higher rates.  There is a tradeoff as higher interrupt rates give more overhead.  This will require some experimentation.

\subsubsection{Terminal Interfaces}
Multiple terminal interfaces can be supported.  Right now, each interface has its own interrupt vector, but it should be possible to have a single vector where the service routine determines which interfaces are ready.

\subsection{Scheduling}
Currently, a simple round-robin scheduler is used.

\subsection{System Calls}
Currently, all system calls are handled by \verb|TRAP #0|.  The system call number and any parameters are pushed onto the stack prior to the call.  The following system calls are currently defined:
\begin{description}
  \item[0 SYS\_EXIT] - Exit program
  \item[1 SYS\_PUTS] - Send a string to the console
  \item[2 SYS\_GETC] - Get a character from the console
  \item[3 SYS\_PUTC] - Send a character to the console
  \item[16 SYS\_SLEEP] - Suspend current task for a number of clock ticks
  \item[64 SYS\_SHUTDOWN] - Shuts the system down
\end{description}

%--------------------------------------------------------------------------------------------------
\section{Library}
The library starts with a table of addresses of the various library routines.  This allows the user programs to find the desired routine by looking in a fixed address.  The code for a library call looks something like (Other registers besides \verb|%A0| can be used, but \verb|%A7| is the stack pointer and \verb|%A6| is often used as a frame pointer.  The macros use \verb|%A0|):
\begin{lstlisting}
...  Put stuff on the stack
    move.l #LIBTBL,%A0
    move.l LIB_GETSTR(%A0),%A0
    jsr (%A0)
...  Cleanup the stack
\end{lstlisting}

Since the library routines can be preempted, the library must contain only reentrant code.

%--------------------------------------------------------------------------------------------------
\section{User Space}
Each task is allocated 1 megabyte of space starting on a megabyte boundary.  The initial PC is the start of the space and the initial SP is the end of the space.  The user program can use this space as it sees fit.

\chapter{Kernel}
This chapter describes the operating system kernel in more detail.
%--------------------------------------------------------------------------------------------------
\section{OS Macros and Definitions}
A number of macros and data structures are defined in order to help keep the kernel code consistent and easier to understand.

\subsection{Definitions}
The data structures are defined using macros to help avoid repeating code and so that changes need only to be made in the macro definition.  In addition to the macro defining the data structure, symbols are defined for various offsets and data within the structure.

The core of the multi-tasking is the task table.  It is basically an array of longword addresses to task control blocks (TCBs).  The table is located by global symbol \verb|TASKTBL|.  The task number of the currently executing task is stored in a word located at global symbol \verb|CURRTASK|.  It is initialized to 1.  The maximum number of tasks defined is another global symbol \verb|MAXTASK|, which is used to identify the end of the task table.
\subsubsection{Task Control Block (TCB)}
Each task has a task control block (TCB) with the following structure:
\begin{lstlisting}
.macro TCB entry,stack,console
    .dc.w 0             | PSW (0)
    .dc.l \entry        | PC  (2)
    .dc.l 0             | D0  (6)
    .dc.l 0             | D1 (10)
    .dc.l 0             | D2 (14)
    .dc.l 0             | D3 (18)
    .dc.l 0             | D4 (22)
    .dc.l 0             | D5 (26)
    .dc.l 0             | D6 (30)
    .dc.l 0             | D7 (34)
    .dc.l 0             | A0 (38)
    .dc.l 0             | A1 (42)
    .dc.l 0             | A2 (46)
    .dc.l 0             | A3 (50)
    .dc.l 0             | A4 (54)
    .dc.l 0             | A5 (58)
    .dc.l 0             | A6 (62)
    .dc.l \stack        | SP (66)
    .dc.l 0             | Task status (70)
    .dc.l 0             | Sleep timer (74)
    .dc.l \console      | Console device (78)
.endm
\end{lstlisting}

The offsets are defined as follows:
\begin{lstlisting}
    .equ TCB_PSW,    0
    .equ TCB_PC,     2
    .equ TCB_D0,     6
    .equ TCB_D1,    10
    .equ TCB_D2,    14
    .equ TCB_D3,    18
    .equ TCB_D4,    22
    .equ TCB_D5,    26
    .equ TCB_D6,    30
    .equ TCB_D7,    34
    .equ TCB_A0,    38
    .equ TCB_A1,    42
    .equ TCB_A2,    46
    .equ TCB_A3,    50
    .equ TCB_A4,    54
    .equ TCB_A5,    58
    .equ TCB_A6,    62
    .equ TCB_SP,    66
    .equ TCB_STAT0, 70
    .equ TCB_STAT1, 71
    .equ TCB_STAT2, 72
    .equ TCB_STAT3, 73
    .equ TCB_SLEEP, 74
    .equ TCB_CON,   78
\end{lstlisting}

The current TCB status flags are defined for \verb|TCB_STAT0|.  The rest of the bits are unused.  The only flags defined now are for types of waiting.  This allows the scheduler to do a simple test for zero on the \verb|TCB_STAT| longword.  If flags ever need to be defined for non-wait purposes, the 32 bit status word may be split into two 16 bit words, with one word describing waits and the other describing non-wait status.  The currently defined flags are:
\begin{lstlisting}
   .equ TCB_FLG_IO,    0
   .equ TCB_FLG_SLEEP, 1
   .equ TCB_FLG_EXIT,  2
\end{lstlisting}
\verb|TCB_FLG_EXIT| is used to indicated that a task has terminated and should no longer be scheduled.  Since a shared command line interpreter is available, this flag may get depricated.  At some point, \verb|TCB_FLG_IO| may be split as more types of I/O get defined.

\subsection{Macros}

The \verb|GET_TCB| macro is defined as follows, and is used whenever a pointer to the current task's TCB is needed:
\begin{lstlisting}
.macro GET_TCB reg
    move.l #0,\reg            |  Ensure that high bits are cleared
    move.w CURRTASK,\reg      |  Get current task number
    add.l \reg,\reg
    add.l \reg,\reg           |  Multiply by 4
    move.l TASKTBL(\reg),\reg |  Index into TCB table
.endm
\end{lstlisting}

%--------------------------------------------------------------------------------------------------
\section{Scheduler}
Scheduling tasks consists of three basic functions.  The first is saving the context of the current task, then determine which task to run next, and finally restoring context for the new task.

The context is saved by the following routine.  It saves register \verb|%A6| and uses it as a pointer to the current task's TCB.  Some values are read off the stack and written into the TCB.  Most of the registers are saved using a \verb|movem.l| instruction
\begin{lstlisting}
CTXSAVE:
    .global CTXSAVE
    MOVE.L %A6,-(%SP)    |  A6 is used to get a pointer to the task data area
    |
    |  At this point, the stack is:
    |   0(SP) -> A6
    |   4(SP) -> PSW
    |   6(SP) -> PC for return
    |
    GET_TCB %A6
    MOVE.W 8(%SP),(%A6)          |  Save PSW
    MOVE.L 10(%SP),TCB_PC(%A6)   |  Save PC
    MOVEM.L %D0-%D7/%A0-%A5,TCB_D0(%A6) |  Most registers are now saved
    MOVE.L %USP,%A0
    MOVE.L (%SP)+,TCB_A6(%A6)    |  Save A6
    MOVE.L %A0,TCB_SP(%A6)       |  Save USP
    move.l TCB_A0(%A6),%A0       |  Restore A0
    move.l TCB_A6(%A6),%A6       |  Restore A6
    RTS
\end{lstlisting}


\end{document}

