\documentclass[10pt]{article}
%
\usepackage{url}
%
%  Include the listings package
%
\usepackage{listings}
%
%  Define Tiny Lisp based on Common Lisp
%
\lstdefinelanguage[Tiny]{Lisp}[]{Lisp}{morekeywords=[13]{atomp, bit-vector-p, car, cdr, char-downcase, char-code, char-upcase, compiled-function-p, dowhile, dump, exit, fresh-line, if, code-char, lambda, msg, nullp, parse-integer, peek8, peek16, peek32, poke8, poke16, poke32, progn, quote, read-line, reset, setq, simple-bit-vector-p, simple-string-p, simple-vector-p, string-downcase, string-upcase}}
\lstset{language={[Motorola68k]Assembler}}
%
% Macro definitions
%
\newcommand{\operation}[1]{\textbf{\texttt{#1}}}
\newcommand{\package}[1]{\texttt{#1}}
\newcommand{\function}[1]{\texttt{#1}}
\newcommand{\constant}[1]{\emph{\texttt{#1}}}
\newcommand{\keyword}[1]{\texttt{#1}}
\newcommand{\datatype}[1]{\texttt{#1}}
\newcommand{\tl}{Tiny-Lisp}
\newcommand{\cl}{Common Lisp}

%
% Front Matter
%
\title{Notes and Background on Operating System for the 68000 CPU}
\author{Brent Seidel \\ Phoenix, AZ}
\date{ \today }
%========================================================
%%% BEGIN DOCUMENT
\begin{document}
\maketitle
\begin{center}
This document is \copyright 2024 Brent Seidel.  All rights reserved.

\paragraph{}Note that this is a draft version and not the final version for publication.
\end{center}
\tableofcontents

%--------------------------------------------------------------------------------------------------
\section{Overview}

This is a collection of notes on a simple multi-tasking operating system for the 68000 CPU, temporarily names os68k.  The main goals are:
\begin{enumerate}
  \item To be able to blink lights in interesting pattern in the Pi-Mainframe (\url{https://github.com/BrentSeidel/Pi-Mainframe}) project,
  \item To learn something about operating system design, and
  \item To actually have a somewhat useful operating system.
\end{enumerate}

Note that all statements are subject to change as the project develops.

This is based on the 68000 simulator part of the Sim-CPU (\url{https://github.com/BrentSeidel/Sim-CPU}) project.  It is targeted towards a system with 16Mbytes of memory and no MMU.  The memory is divided into 16 one megabyte sections, the first (lowest) section is for the operating system and each task gets one section.  The gives a system that supports up to 15 user tasks plus the operating system.

Simulated devices provided by Sim-CPU that are currently used are the clock to provide a periodic interrupt for tasking and the serial-telnet port for console I/O.  Simulated disks are not yet supported.

The operating system also includes a library with utility routines that can be used by the user programs.

\section{Kernel}
The kernel is composed of several sections.  The first two are at locations that are fixed by the simulated CPU and hardware:
\begin{enumerate}
  \item The CPU vector table starts at address 0 and contains 256 long word entries.  This occupies 1 kilobyte of space.
  \item The I/O port section runs from the end of the vector table at an address of 400$_{16}$ and runs to 1000$_{16}$.
\end{enumerate}
The remaining sections are arbitrarily arranged and will probably change with development.
\begin{enumerate}
  \item HW\_SECT contains code for interfacing with the hardware devices and related interrupt service routines.
  \item OS\_SECT contains the main operating system code including initialization, the clock interrupt, context save/restore, scheduling, exceptions, and system calls.
  \item OS\_DATA contains the operating system data tables and messages.  The data tables currently include the task control blocks and the console device blocks.
  \item LIB\_SECT contains library routines for use by the operating system and user programs.  Note that routines that use system calls shouldn't be used by the operating system, at least not yet.
  \item LIB\_DATA contains library data.  Note that since library routines may be in use by multiple tasks simultaneously, this should be constant data only.  Any variable data should be allocated on the stack.
  \item OS\_STACK is space for the operating system stack.
  \item USR\_STACK is space for the operating system user stack.  The is used by the null task that runs when no other task can be run.  Its stack needs are minimal.
\end{enumerate}

\subsection{Hardware Abstraction Layer}
The currently supported hardware includes a clock that provides periodic interrupts and a terminal interface that can be accessed externally by telnet (or gtelnet).

\subsubsection{Clock}
The clock has a settable rate for the periodic interrupts.  The rate is a byte size port which allows any value 0-255.  The multiplier in the simulation is 100mS, thus to get a 1 second interrupt, the rate would be set to 10.  Setting the rate to one gives a 100mS interrupt (or 10 times a second).  This is a reasonable rate for multitasking.  As some point, the simulation may get adjusted to allow higher rates.  There is a tradeoff as higher interrupt rates give more overhead.  This will require some experimentation.

\subsubsection{Terminal Interfaces}
Multiple terminal interfaces can be supported.  Right now, each interface has its own interrupt vector, but it should be possible to have a single vector where the service routine determines which interfaces are ready.

\subsection{Scheduling}
Currently, a simple round-robin scheduler is used.

\subsection{System Calls}
Currently, all system calls are handled by \verb|TRAP #0|.  The system call number and any parameters are pushed onto the stack prior to the call.  The following system calls are currently defined:
\begin{description}
  \item[0 SYS\_EXIT] - Exit program
  \item[1 SYS\_PUTS] - Send a string to the console
  \item[2 SYS\_GETC] - Get a character from the console
  \item[3 SYS\_PUTC] - Send a character to the console
  \item[16 SYS\_SLEEP] - Suspend current task for a number of clock ticks
  \item[64 SYS\_SHUTDOWN] - Shuts the system down
\end{description}

\section{Library}
The library starts with a table of addresses of the various library routines.  This allows the user programs to find the desired routine by looking in a fixed address.  The code for a library call looks something like (Other registers besides \%A0 can be used, but \%A7 is the stack pointer and \%A6 is often used as a frame pointer.  The macros use \%A0):
\begin{lstlisting}
...  Put stuff on the stack
    move.l #LIBTBL,%A0
    move.l LIB_GETSTR(%A0),%A0
    jsr (%A0)
...  Cleanup the stack
\end{lstlisting}


Since the library routines can be preempted, the library must contain only reentrant code.

\section{User Space}
Each task is allocated 1 megabyte of space starting on a megabyte boundary.  The initial PC is the start of the space and the initial SP is the end of the space.  The user program can use this space as it sees fit.

\end{document}

